% !TeX program = pdfLaTeX
\documentclass[12pt]{article}
\usepackage{amsmath}
\usepackage{graphicx,psfrag,epsf}
\usepackage{enumerate}
\usepackage{natbib}
\usepackage{url} % not crucial - just used below for the URL

%\pdfminorversion=4
% NOTE: To produce blinded version, replace "0" with "1" below.
\newcommand{\blind}{0}

% DON'T change margins - should be 1 inch all around.
\addtolength{\oddsidemargin}{-.5in}%
\addtolength{\evensidemargin}{-.5in}%
\addtolength{\textwidth}{1in}%
\addtolength{\textheight}{1.3in}%
\addtolength{\topmargin}{-.8in}%

\begin{document}

\def\spacingset#1{\renewcommand{\baselinestretch}%
{#1}\small\normalsize} \spacingset{1}


%%%%%%%%%%%%%%%%%%%%%%%%%%%%%%%%%%%%%%%%%%%%%%%%%%%%%%%%%%%%%%%%%%%%%%%%%%%%%%

\if0\blind
{
  \title{\bf MDM Microservices}

  \author{
        Gurpreet Singh \thanks{The author gratefully acknowledges \ldots{} David Song, Simon Kotwicz,
and Herman Singh} \\
    Infosphere MDM Delivery, IBM\\
      }
  \maketitle
} \fi

\if1\blind
{
  \bigskip
  \bigskip
  \bigskip
  \begin{center}
    {\LARGE\bf MDM Microservices}
  \end{center}
  \medskip
} \fi

\bigskip
\begin{abstract}
A proposal to migrate existing tools and develop new tools using a
microservice based architecture. A microservice based approach provides
solutions to problems regarding property management, user friendliness,
and many internal development issues. A proof of concept architecture
implementation is attached and documented to show the potential benefits
to the team.
\end{abstract}

\noindent%
{\it Keywords:} Bluemix, Docker, Golang, Kotlin, Javascript, rMarkdown
\vfill

\newpage
\spacingset{1.45} % DON'T change the spacing!

\def\tightlist{}

\section{Problem Analysis}\label{problem-analysis}

\subsection{High Level Goals}\label{high-level-goals}

My interpertation of the contest requirements are summerized into the
following 3 points

\begin{itemize}
\item
  More user friendly tooling
\item
  Easier for tools to access common properties
\item
  Elegent solution for property management
\end{itemize}

\subsection{Unrealized Problems}\label{unrealized-problems}

The following is a list of problems I realized throughout the past year
using the current set of tools.

\begin{itemize}
\item
  Tools unable to communicate with each other
\item
  Incredibly difficult and complex learning curve to reach 1 successful
  run of any tool

  \begin{itemize}
  \tightlist
  \item
    Too many tutorials to follow, too many settings to change just to
    run a simple test
  \end{itemize}
\item
  Language and platform restrictions becoming larger with each new tool

  \begin{itemize}
  \tightlist
  \item
    Restricting possible approaches to a new problem
  \end{itemize}
\item
  High overhead during initial setup phase for every tool

  \begin{itemize}
  \item
    Only a few people know how to set everything up
  \item
    Leads to too many environment problems that should never occur for a
    user
  \end{itemize}
\item
  Current programming paterns make it hard to implement test driven
  development
\end{itemize}

\section{Solution}\label{solution}

\subsubsection{Microservices}\label{microservices}

\emph{Microservices architecture is an approach to application
development in which a large application is built as a suite of modular
services. Each module supports a specific business goal and uses a
simple, well-defined interface to communicate with other sets of
services.} \citep{searchMicroservices}

\begin{itemize}
\item
  Everything is a microservice

  \begin{itemize}
  \tightlist
  \item
    Tools handle one specific task, very efficently
  \end{itemize}
\item
  Tools are developed and run out of Docker containers
\item
  Tools are setup on central machines and run through an interface (web
  / CLI / anything)

  \begin{itemize}
  \tightlist
  \item
    This means one time setup and maintainence, no more setup issues.
  \end{itemize}
\item
  Inter-tool communication is done through HTTP

  \begin{itemize}
  \item
    HTTP is easily accessible from any platform and any programming
    language
  \item
    Using a series of HTTP methods (GET, PUT, POST, DELETE\ldots{})
    anything can use a tool
  \end{itemize}
\item
  Every tool will have a set of API endpoints that clearly define that
  tool's usage

  \begin{itemize}
  \item
    We could have one tool written in Java, another tool written in
    Pythong and they could both work together without any modification.
  \item
    A clear API will enforce and encourage clear tests
  \end{itemize}
\item
  A set of endpoints will be required for any tool to work with the
  architecture

  \begin{itemize}
  \item
    Each tool will register on bootup with the Core service providing
    its required property set
  \item
    Provides a /status and /run endpoint for UI interactions
  \item
    Informs Core when a run has finished
  \end{itemize}
\item
  Property values will be stored on the Core service

  \begin{itemize}
  \tightlist
  \item
    Each tool will use the /fetch endpoint and send in a run id to
    retireve the properties it needs
  \end{itemize}
\end{itemize}

\section{Functional Requirements}\label{functional-requirements}

\begin{verbatim}
Let an interface be the WebUI or any other user friendly frontend able 
to call HTTP methods to a service

Let a tool be the Tool or any other delivery tooling that does work and is 
able to call HTTP methods to a service
\end{verbatim}

\subsection{Core}\label{core}

\begin{itemize}
\item
  Allows new services to register. Stores their properties.
\item
  Allows an interface to fetch list of services
\item
  Can check the status of any service
\item
  Allows an interface to create a new run with any predefined property
  set
\item
  Queues up requests and load balences them between services if there
  are multiple instances of a tool running
\item
  Allows a tool to submit a run result
\item
  Allows an interface to fetch a run result
\end{itemize}

\subsection{A Tool}\label{a-tool}

\begin{itemize}
\item
  Registers with the Core at the start of the service and sends it a
  list of required properties
\item
  Able to provide it's status, even in the middle of a run
\item
  Fetches new properties from Core for every run
\item
  Sends a report to Core after every run
\end{itemize}

\subsection{WebUI}\label{webui}

\begin{itemize}
\item
  Gets a list of services from Core
\item
  Creates a run with one of the property maps for one of the services
  retrieved
\item
  Checks back for the result of the run and displays the report
\end{itemize}

\bibliographystyle{agsm}
\bibliography{bibliography.bib}

\end{document}
